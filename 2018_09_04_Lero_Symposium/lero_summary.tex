% 
% Author: Riccardo Orizio (R)
% Date: Mon 13 Aug 2018
% 
% Description: Summary proposal for Lero Symposium to be held on September 4th
% 2018
%

\documentclass{article}
\usepackage{graphicx}
\pagenumbering{gobble}

% Title contents
\title
{
	\vspace{-2cm}
	%	A data-driven approach to diagnosis
	Improving security and resilience of Cyber Physical Systems
}
\author{Riccardo Orizio}
\date{}

\begin{document}

\maketitle

Can we create a tool that can help Cyber Physical systems in detecting,
identifying and correcting an anomaly whenever one would occur?
Can it be used for time critical systems?

Our research method is experimental based.
Currently we are working with simulation data but we hope to use some data from
real systems scenarios.

So far these are the techniques that we were able to test: pure model based
approach; a basic residual study approach; a basic and an improved
version of the algebraic approach; data driven approaches.
All of these techniques have been tested on simulated data and their results
have been compared with each other trying to identify what can be considered the
best approach, when possible.

In the near future we want to identify what are the best approaches to use in
the diagnosis process on our simulation example and then extending it to a real
data example.
We are aiming in finding a set of data driven approaches which, alone or
combined, can increase the diagnosis process performaces.

I started my PhD course in April 2017 as a full time student in University
College Cork under the supervision of professor Gregory Provan.

%	%	1. Research Question
%	Cyber Physical Systems are increasing in number and their security is of main
%	importance, especially in process-control and SCADA systems.
%	These systems have to be able to promptly react to any kind of anomaly that
%	affects them, either these anomalies are genuine or malicious
%	(e.g. respectively, a fault on a component of the system or an external attack
%	 aimed to disrupt some functionalities of the system).
%	Can we create a tool that can help these systems in detecting, identifying and
%	correcting an anomaly whenever one would occur?
%	Can it be used for real time systems?
%	And can it be an all purpose tool used on a wide variety of different systems?
%	
%	%	2. Research Method
%	Our research method is experimental based.
%	Currently we are working with simulation data but we hope to include some data
%	from real test case scenarios.
%	%	Starting from simple examples and with their simulation data, we want to test if
%	%	performing diagnosis on a system with a data driven approach can be more
%	%	efficient and effective than a model based approach.
%	
%	%	3. Summary of Research to date
%	So far these are the techniques that we were able to test: pure model based
%	approach; a basic residual study approach; a basic and a personally improved
%	version of the algebraic approach; data driven approaches, such as SVM and LSTM.
%	All of these techniques have been tested on simulated data and their results
%	have been compared with each other trying to identify what can be considered the
%	best approach, when possible.
%	
%	%	4. Plan for next 12 months
%	In the near future we want to identify what are the best approaches to use in
%	the diagnosis process on our simulation example.
%	We are aiming in finding a set of data driven only approaches given their
%	reduced hardware requirements compared to what is required by model based
%	approaches.
%	We will then work on various ways to find better diagnosis results combining
%	these approaches together instead of having them work separately.
%	Afterwards we would like to test our method on real data and see its
%	performance.
%	
%	%	Produce an all purpose tool that, after a small training period, is able to
%	%	improve the security of the current system, helping the anomaly identification
%	%	process on the system and generating a real time answer able to limit its
%	%	consequences.
%	%	The tool will be based on a set of different data driven techniques proven to be
%	%	as efficient as a model based approach while being less demanding on the
%	%	hardware requirements of the system.
%	
%	%	5. Other details: How long you have been registered as a student?
%	%	Full-time/Part-time? Expected completion date. University in which you are
%	%	registered. Supervisor(s).
%	I started my PhD course in April 2017 as a full time student in University
%	College Cork under the supervision of professor Gregory Provan.

\end{document}


